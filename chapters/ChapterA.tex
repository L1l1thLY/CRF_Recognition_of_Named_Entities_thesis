% !Mode:: "TeX:UTF-8"

\chapter{绪论}
\section{研究工作的背景与意义}
纵观人类历史,语言是人类文明发展的载体。语言以及与其对应的文字为人类特有的用来传递信息的符号系统,是人类与动物在智能方面最具决定性的差异。这里所描述的语言,是指人类发展自然形成的语言,如日语、汉语等。和人类为某种目的创造的语言(比如编程语言)不同,人们通常称其为自然语言。

自然语言处理是计算机科学领域的一个重要方向,它所研究的就是计算机与人类之间使用自然语言交流的理论和方法。众所周知,科技、文学和宗教,人类能够理解的信息几乎都通过自然语言传承和传播。在现代社会,甚至人类的逻辑与思考都是以语言为基础的。由此可见此学科方向的实际意义十分显著,如果计算机能够理解人类自然语言的特殊含义,那么就能让计算机帮助人类完成一些需要大量人力的工作,比如翻译工作,对互联网上的大量自然语言文本进行分析等。而人类操作计算机工作的方式也不再以学习繁杂的、不符合人类习惯的计算机语言为前提。而在研究计算机理解人类语言的过程中,也能增进我们对人类是如何理解语言这里问题的认识。

为能够实现人类与计算机的通信,研究既要完成计算机理解自然语言文本意义的目标,也要完成计算机使用自然语言表达特定的含义的目标。前者称为自然语言理解,后者称为自然语言生成。

中文文本形式上可以看做是由汉字及标点符号组成的字符序列。含义通常由完整的句子表达,而句子是由词组组成的,词组又由单个汉字组成。但无论是任何层次都存在着歧义和多义的情况,举个例子,在某个句子中的词组,可能在另一个句子中含义大相径庭,同时,两个完全不同的句子中两个不同的单词也有可能有相同的含义。这给自然语言理解带来了最直接的困难。但实际上,自然语言通常是不存在歧义的,这时由于我们在交流中,大脑通常会结合特定的上下文和经验,得到准确的含义。如何让计算机获得理解上下文的能力和人类特有的经验便是自然语言理解的主要工作。

现代的自然语言处理工作通常是基于机器学习,通常是统计机器学习。
\section{命名实体的概念以及其地位}
命名实体指的是文本中具有特殊意义的词语、短语。比如专有名词、人名、地名、机构名等,有时也根据需求包括时间、数量等。根据不同的应用场景,命名实体的确切含义则会有所不同。由此可见命名实体通常是一串文本中最基本、最重要的信息元素,是理解文本的基础。

至今,命名实体的研究越来越受到重视。1995年9月举行的MUC-6会议首次定义了“命名实体”这一术语,同时提出了一个新的领域,此领域旨在对英文命名实体的结果进行评测。目光转回国内,863计划中文信息处理与智能人机交互技术评测会议将中文命名实体识别作为分词和词性标注的子任务引入。中文与英文的差距较大,具有特殊性,便与命名实体的开放性和发展性产生了矛盾,导致中文命名实体研究进展较为缓慢。  

初等数学中,概率和统计在我国数学高考试卷中占据着重要地位,其贴近生活而情景多变,以应用题为主。对于此类问题的计算机自动类人解题充满了挑战性。若将解题过程分为多个过程,最开始的过程便是题目的形式化表征。通俗来讲,是如何才能让计算机理解自然语言编写的题干含义的问题。自然语言和数学语言是存在差距的,而连接他们的桥梁便是问题的表征。问题表征即是挖掘问题中元素和元素间的关系,而这些元素以及元素间的关系将成为计算机自动类人解题过程中用于求解模型的参数。也就是说,我们创造的问题形式化表征生成系统,首先应拥有搜索题目中的元素以及元素之间约束关系的能力。结合上面提到的实体的定义,在这个应用场景下,题目中的元素以及可能的元素约束关系,将是文本中最具有特殊意义也最首要的信息,也就是所谓的命名实体。  
\section{本文的主要贡献与创新}
本论文以时域积分方程时间步进算法的数值实现技术、后时稳定性问题以及
两层平面波加速算法为重点研究内容,主要创新点与贡献如下:

……
\section{本论文的结构安排}
本文的章节结构安排如下:

……
