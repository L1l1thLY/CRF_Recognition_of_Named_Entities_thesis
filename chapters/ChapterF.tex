% !Mode:: "TeX:UTF-8"

\chapter{总结与展望}
\section{总结}
本文在预设的初等数学概率与统计题自动求解工作的背景下,分析得出解题的首要工作为计算机对数学题意的语义理解。对于语义的理解的关键在于关键信息的提取,而对于初等数学概率与统计题,命名实体即使其中最关键的信息点。

本文对于多个可以用于命名实体识别的概率图模型进行了简要的描述和比较,最终确定了条件随机场对于此工作的适用性和优越性。本文的创新点如下:

(1)以自动解题目的为指导下对于初等数学概率与统计题目中数学命名实体的确定。本文结合实例,分析了初等数学概率与统计题的解题过程,结合解题过程,确定了初等数题解题过程中所需的关键条件为一些命名,及与命名对应的数字与单位。并确定了为了能够达到解题,需要确定的关键条件。并由此确定了所需的标注集合为名字实体,数字实体以及单位实体。

(2)基于CRF原理的适用于中文初等数学命名实体标注的特征研究:首先充分地分析了条件随机场模型中特征函数的含义以及CRF++中特征模板生成特征函数的规律后,根据中文初等数学命名实体的词性,词形以及组合多词的方法构建了组合特征。提升了识别效果。

\section{展望}
本文通过人工标注命名实体语料数据训练概率模型的方法来实现识别初等数学命名实体。而单层条件随机场的识别效果有限,尤其是初等数学名字实体构词规则更复杂,识别效果较差。我们认为,结合一些构词规则对未识别出的实体进行二次识别\citeup{pan2012guizetongji1},或使用层叠条件随机场可能会有更好的效果\citeup{zhou2006cengdiecrf}。在今后的研究工作,会考虑借鉴这些方法。而分类机器学习模对于数据的人工标注工作量巨大,而且不可能将所有数据加入训练集合,今后可能考虑研究半聚类的算法,以减轻人工工作。