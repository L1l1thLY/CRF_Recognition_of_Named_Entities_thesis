% !Mode:: "TeX:UTF-8"

\begin{Cabstract}{自然语言处理}{中文命名实体识别}{条件随机场}{初等数学}{}
自然语言是人类文明的载体,随着计算机科学的发展,人工智能技术的研究方向自然而然聚焦在计算机与人类之间通信的可行性。对于命名实体的识别成为了自然语言处理领域的重要课题之一、至今为止,其研究愈发受到重视。而在其概念之上,知识推理的概念被提出,知识推理是机器学习、深度学习研究又是最重要、最核心的问题。因此基于知识推理的863课题“类人求解系统”被提出。

以CRF(条件随机场)模型识别为基础,实现了对于初等数学文本中数学命名实体信息的标注。选择了合适于数学实体标注的标志特征作为系统训练的的特征集合。结合前人的经验,选择了较好的特征模板,验证并分析了在CRF模型中使用的特征的有效性。

本文结合理论和实践两方面重点研究了如何正确高效抽取初等数学概率与统计题中关键信息的方法,主要进行了以下几个方面的研究:

1、对于可用于命名实体标注的相关算法研究

对于中文命名实体识别,世界上已经有很多成功的实践,其应用通常基于概率图模型。本文对于已经应用于中文命名实体识别的一些概率模型进行了研究,并作出一些对比。

2、初等数学概率统计题自动求解场合下的命名实体标注研究:

首先分析了初等数学概率统计题的语言特点,依据实际的解题过程确定了初等数学概率统计题需要的命名实体标注集合,有实际意义。

3、基于CRF的初等数学命名实体识别算法:

CRF算法在应用时应规定特征函数,本文对特征函数进行了分析,并由于没有相关的应用于解题的数学实体研究,本文从词性、词形出发设计了原子特征与组合。最终构建了一个基于CRF的初等数学概率与统计问题命名实体识别系统。
\end{Cabstract}
