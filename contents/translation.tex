% !Mode:: "TeX:UTF-8"

随机数据有两个特点:第一,我们希望能够标注的实体之间存在统计上的依赖性。第二,每一个实体通常有丰富的特征来辅助分类。例如。当我们分类web文件时,网页中的文本提供了关于分类标签的大量信息,但是超链接定义了网页之间的联系,这可以优化分类
图模型是一种利用实体间的依赖结构的自然形式。传统上,图模型用于表示联合分布律p(x,y),其中变量y表示我们希望预测的实体的属性,输入变量x表示我们观测到的实体的特征。但是在应用大量的关联数据的局部特征进行联合概率建模会有困难,因为这要求对分布p(x)建模,其中包含了纷杂的依赖性。在输入中对这些依赖性建模会产生难以控制的模型,但是忽略他们会降低效果。解决这个问题就是直接对条件概率p(y|x)建模,能够进行充分的分类。这是crf一文中提出的方法。条件随机场是一个简单的条件概率p(y|x)结合了一个相关的图结构。因为模型是随机的,输入变量x之间的依赖性并不需要被精确地表示出来,提供了输入的全局特征使用。例如,在自然语言处理任务中,有用的特征包括邻近的词、二元语法、前缀、后缀、在特定领域里的词汇、以及像来源于Wordnet的语义信息。最近,随着在文本、生物信息学以及计算机视觉等领域的成功应用,人们对于CRF的兴致越来越为浓厚。

这一章分为两个部分:

第一部分中我们提出了目前训练的说明以及条件随机场的推理技术。我们探讨重要的线性链条件随机场特例,并将其推广到任意的图结构上。也包括一段简短的讨论关于实际CRF应用所需的技术。

第二部分中我们提供了一个将综合CRF应用于关系学习问题中的例子。特别的,我们探讨了信息抽取的问题,即自动的从无结构的语料中建立有关联的数据库。与线性链CRF不同,综合CRF能得到标签之间远距离的依赖性例如,如果同样的名字在一个文件中提到了超过一次,所有的提及可能会获得相同的标签,并且这对于对他们全部的提取很有用处,因为每次提及也许会包涵关于潜在实体不同的互补的信息。为了实现这种远距离依赖性,我们拟采用跳跃链CRF,它会同时对提到的内容进行分隔与集合的标注。在一个标准的从研讨会声明提取发言者名字的问题中,跳跃链CRF相比于线性链CRF会有更好的表现。